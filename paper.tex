\documentclass{article}
\usepackage{url,color,xspace,verbatim,subfig,ctable,multirow,listings}
\usepackage[utf8]{inputenc}
\usepackage[T1]{fontenc}
\usepackage{txfonts}
\usepackage{rotating}
\usepackage{paralist}
\usepackage{subfig}
\usepackage{graphics}
\usepackage{enumitem}
\usepackage{times}
\usepackage{amssymb}
\usepackage[colorlinks=true]{hyperref}

% ==================================================

\graphicspath{{figs/}}
\urlstyle{sf}

% tikz stuff
\usepackage{tikz}
\usepackage{pgfplots}
% configuration
\usetikzlibrary{shapes,positioning,calc,snakes,arrows,shapes}
\pgfplotsset{width=.9\linewidth}

\lstset{
  language=C,
  basicstyle=\ttfamily \small,
  flexiblecolumns=false,
  basewidth={0.5em,0.45em},
  boxpos=t,
}

\definecolor{skRed}{RGB}{155,25,25}
\newcommand{\stefan}[1]{
  {\color{skRed}[{\color{red}{SK}} #1]}}

% ==================================================
\setcounter{section}{0} % Start sections with 1, not 0
\begin{document}

\title{Replication on multicore considering hierarchy and
  characteristics of the network}

% email address
\newcommand{\eaddr}{stefan.kaestle@inf.ethz.ch}
\newcommand{\email}{\href{mailto:\eaddr}{\eaddr}}

\author{Stefan Kaestle\\
  \email \\
  Systems Group, ETH Zurich}

\maketitle

%%%%%%%%%%%%%%%%%%%%%%%%%%%%%%%%%%%%%%%%%%%%%%%%%%
\section{Introduction}

\paragraph{What do you want to enable?} Increase algorithm performance
on multicores by applying distributed principles to overcome
scalability challenges on such machines. 

\paragraph{What problem are you solving, and why is it hard?} We are
applying distributed algorithms to multicores to enable better
scalability. However, characteristics of a multicore a different to
those of traditional distributed systems. Traditional algorithms need
to be reevaluated for multicores.

\paragraph{What's the related work?} People apply techniques from
distributed systems to deal with scalability challenges on multicores,
e.g.\ databases~\cite{Salomie2011, Wiesmann2000} and operating
systems~\cite{fos:osr09, tornado:osdi99, barrelfish:sosp09}. But they
don't to it properly. They do not consider machine characteristics or
apply the wrong distributed technologies. Also, what they do is rather
ad hoc. 

\paragraph{What new idea(s) will solve the problems?} Instead of just
reusing distributed algorithms, we reevaluate them for multicores and
investigate the impact of the multicore network model on algorithm
performance.

\paragraph{How will you go about it?} We investigate the problem for
atomic-broadcasts since many problems in distributed systems can be
reduced to this communication primitive. Traditional implementations
are sending requests to all nodes in parallel to hide high propagation
delays. %
However, this does not work on multicores, since the propagation time
is not dominating any longer. %
Instead, we build a tree for communication to parallelize transmit and
receive cost across nodes. This has been done before (BF SOSP, IBM,
.. ), but not for the general case of multicore communication. 

\paragraph{How will you know and show it works?} We will use a
microbenchmark measuring broadcast cost and compare our tree
implementation to the sequential case. Maybe we put some high-level
applications on top of it.

\paragraph{What is your hypothesis?} Performance of distributed
algorithms can be improved if characteristics of multicores are
considered when building and/or selecting them. %
For the broadcast example, this means: the time complexity for a
networks with dominating transmit and receive time will be $O(log(n))$
instead of $O(n)$.

%%%%%%%%%%%%%%%%%%%%%%%%%%%%%%%%%%%%%%%%%%%%%%%%%%
\section{Related work}

A multicore aware broadcast has been implemented in~\cite{Tu2008}. It
takes the cache architecture of the machine into consideration for
building the broadcast. It is however for larger messages (starting at
4K) and it does not seem to be performing very well (numbers are given
in microseconds -6). 4K messages are 52 microseconds, which is 150000
cycles. For synchronization primitives, a cache line is sufficient.

Another tree implementation with MPI in mind was done
in~\cite{Graham2008}. They have comparable numbers to what we found. 

%%%%%%%%%%%%%%%%%%%%%%%%%%%%%%%%%%%%%%%%%%%%%%%%%%
\section{Model}

Characteristics of the network influence the performance of
distributed algorithms. Multicores networks are heterogeneous
combinations of complex networks, which makes algorithm selection
hard. %
Furthermore, their diversity makes manual selection tedious. Hence, we
aim for an automatic approach for selecting algorithms for a specific
network. %
We express the network characteristics in a multicore network
model. This model at this stage is by no means complete. We are going to
extend it over time to address more complexities of current and future
multicores.

The network model is defined by communication channels, which in turn
are defined by a set of attributes (here sorted by priority):
\begin{itemize}
\item latency (quantitative) between each pair of nodes
\item ``breakdown'' (quantitative) of message send cost: Ethernet has
  dominating propagation time vs.\ dominating send and receive time on
  multicores
\item synchronized clocks: sychronized clocks + bound on propagation
  time $\rightarrow$ \emph{synchronous network} model. \\Then:
  $\exists$ algorithm for synchronous networks $\rightarrow
  \exists$ algorithm for asynchronous
\item reliability (loss, link failure, in-order delivery, phantoms,
  corruption, duplicated messages): Ethernet is not reliable,
  multicore interconnect networks often are
\item Security \& and failure detection (and the cost for that)
\end{itemize}

% Model given
We assume that the model is given (if it is not, we can do online
measurements and read some specs to find the relevant information). We
just define one ourselves for now or build one manually for a
particular machine. In a real-world scenario, this does not work. The
model needs to be created automatically for a particular machine (due
to the diversity of machines, and also the pace at which hardware changes)

% Express as:
We express the model in a graph, where nodes are processors and edges
are (on-hop?) communication links. The machines we consider for now
only have bi-directional communication links. Other machines (like the
SCC) have uni-directional links, which is why we model the graph with
directed graph.

% How to deduct tree from model?
We then use this graph to build a group-communication
tree. Interesting techniques could be: clustering, to find closely
connected subsets, distributed messages across cluster first, and then
within cluster. Maybe also a spanning-tree protocol, to find a minimal
graph such that all nodes can be reached.
% Tree -> scheduling
Assuming the tree, there is a scheduling problems for the order in
which to send messages to children. The intuition is to send on
long-delay links first to hide the more expensive latencies.

% Assumptions
We assume that the cost of sending a message to the same node is 0,
which is not true, since even using a highly optimized implementation
for local communication (like LMP) has a non-negligible cost.

% Limitations
What we do not consider yet is link congestion. 

%%%%%%%%%%%%%%%%%%%%%%%%%%%%%%%%%%%%%%%%%%%%%%%%%%
\section{Mechanisms}

\subsection{Aggregation}

Similarly to what has been done in wireless sensor networks (where it
also matters to reduce the number of messages, but other reasons:
power consumption), we can do aggregation in nodes. In difference to
traditional distributed systems, this works, because it is easy to
deploy custom software on every node in the network. Classical
distributed systems do not typically allow this. Furthermore, reducing
the number of messages at the price of higher complexity does not make
sense there.

Examples for aggregation: number of nodes agreeing to something, find
capabilities (concatenate core ids). 

%%%%%%%%%%%%%%%%%%%%%%%%%%%%%%%%%%%%%%%%%%%%%%%%%%
\section{Implementation}

We implement our quorum protocol on Barrelfish using the UMP
interconnect driver. UMP provides channels between
endpoints. Communication is reliable, and messages are delivered
in-order within a channel. Although UMP channels are uni-directional,
Barrelfish automatically established two uni-directional channels on
bind time. Hence, we consider them as being bi-directional. Every UMP
channels has independent buffers associated with it.

First, we establish connections such that we get a fully connected
graph with vertices being the quorum instances and edges being UMP
channels between them. For that, we need $O(n^2)$ UMP channels. 

We build up UMP channels in rounds $r_i$. The quorum instance on core
$i$ is creating a UMP channel to all nodes $j$ with $j>i$ in
$r_i$. The beginning of a round needs to be synchronized such that
node $i+1$ only starts establishing connections when all other nodes
completed $i$. Synchronization is required since UMP channels are
scheduled round-robin. Messages on different channels are received
non-deterministically. 

Build a tree. Every node has parents and children (except for the root
and the leafs). The number of children does not really matter for the
algorithm. In the init phase, we construct the tree (e.g. broadcast:
one root node, all other nodes are children of the root node,
multicast: every node has two children, NUMA tree: two layers, two
different implementations?

We show an example broadcast tree for a 8x4x1 AMD Barcelona machine
(gruyere) in Figure~\ref{fig:qrm_tree_gruyere}.

\begin{figure}
  % Start of code
% \begin{tikzpicture}[anchor=mid,>=latex',line join=bevel,]
\begin{tikzpicture}[>=latex',line join=bevel,scale=.35]
  \pgfsetlinewidth{1bp}
%%
\pgfsetcolor{black}
  % Edge: 15 -> 31
  \draw [->] (27bp,74.327bp) .. controls (27bp,66.264bp) and (27bp,56.648bp)  .. (27bp,37.667bp);
  % Edge: 9 -> 19
  \draw [->] (459bp,148.33bp) .. controls (459bp,140.26bp) and (459bp,130.65bp)  .. (459bp,111.67bp);
  % Edge: 14 -> 30
  \draw [->] (1050.3bp,151.31bp) .. controls (1060.4bp,140.94bp) and (1073.7bp,127.18bp)  .. (1092bp,108.42bp);
  % Edge: 14 -> 29
  \draw [->] (1035bp,148.33bp) .. controls (1035bp,140.26bp) and (1035bp,130.65bp)  .. (1035bp,111.67bp);
  % Edge: 11 -> 23
  \draw [->] (659.73bp,151.31bp) .. controls (649.64bp,140.94bp) and (636.25bp,127.18bp)  .. (618bp,108.42bp);
  % Edge: 7 -> 15
  \draw [->] (83.731bp,151.31bp) .. controls (73.64bp,140.94bp) and (60.253bp,127.18bp)  .. (42bp,108.42bp);
  % Edge: 0 -> 2
  \draw [->] (588.87bp,377.76bp) .. controls (613.22bp,365.25bp) and (652.76bp,344.93bp)  .. (689.4bp,326.1bp);
  % Edge: 10 -> 22
  \draw [->] (387bp,148.33bp) .. controls (387bp,140.26bp) and (387bp,130.65bp)  .. (387bp,111.67bp);
  % Edge: 13 -> 28
  \draw [->] (935.71bp,149.09bp) .. controls (939.96bp,140.36bp) and (945.15bp,129.68bp)  .. (954.37bp,110.75bp);
  % Edge: 1 -> 3
  \draw [->] (398.64bp,306.66bp) .. controls (360.05bp,293.43bp) and (284.76bp,267.64bp)  .. (231.27bp,249.31bp);
  % Edge: 6 -> 14
  \draw [->] (946.31bp,227.77bp) .. controls (963.36bp,216.09bp) and (988.4bp,198.93bp)  .. (1015.9bp,180.12bp);
  % Edge: 3 -> 8
  \draw [->] (207bp,222.33bp) .. controls (207bp,214.26bp) and (207bp,204.65bp)  .. (207bp,185.67bp);
  % Edge: 13 -> 27
  \draw [->] (918.29bp,149.09bp) .. controls (914.04bp,140.36bp) and (908.85bp,129.68bp)  .. (899.63bp,110.75bp);
  % Edge: 10 -> 21
  \draw [->] (371.73bp,151.31bp) .. controls (361.64bp,140.94bp) and (348.25bp,127.18bp)  .. (330bp,108.42bp);
  % Edge: 12 -> 26
  \draw [->] (762.27bp,151.31bp) .. controls (772.36bp,140.94bp) and (785.75bp,127.18bp)  .. (804bp,108.42bp);
  % Edge: 3 -> 7
  \draw [->] (187.69bp,227.77bp) .. controls (170.64bp,216.09bp) and (145.6bp,198.93bp)  .. (118.14bp,180.12bp);
  % Edge: 5 -> 12
  \draw [->] (719.71bp,223.09bp) .. controls (723.96bp,214.36bp) and (729.15bp,203.68bp)  .. (738.37bp,184.75bp);
  % Edge: 9 -> 20
  \draw [->] (474.27bp,151.31bp) .. controls (484.36bp,140.94bp) and (497.75bp,127.18bp)  .. (516bp,108.42bp);
  % Edge: 1 -> 4
  \draw [->] (423bp,296.33bp) .. controls (423bp,288.26bp) and (423bp,278.65bp)  .. (423bp,259.67bp);
  % Edge: 0 -> 1
  \draw [->] (545.13bp,377.76bp) .. controls (520.78bp,365.25bp) and (481.24bp,344.93bp)  .. (444.6bp,326.1bp);
  % Edge: 2 -> 5
  \draw [->] (711bp,296.33bp) .. controls (711bp,288.26bp) and (711bp,278.65bp)  .. (711bp,259.67bp);
  % Edge: 11 -> 24
  \draw [->] (675bp,148.33bp) .. controls (675bp,140.26bp) and (675bp,130.65bp)  .. (675bp,111.67bp);
  % Edge: 4 -> 10
  \draw [->] (414.29bp,223.09bp) .. controls (410.04bp,214.36bp) and (404.85bp,203.68bp)  .. (395.63bp,184.75bp);
  % Edge: 12 -> 25
  \draw [->] (747bp,148.33bp) .. controls (747bp,140.26bp) and (747bp,130.65bp)  .. (747bp,111.67bp);
  % Edge: 8 -> 17
  \draw [->] (198.29bp,149.09bp) .. controls (194.04bp,140.36bp) and (188.85bp,129.68bp)  .. (179.63bp,110.75bp);
  % Edge: 6 -> 13
  \draw [->] (927bp,222.33bp) .. controls (927bp,214.26bp) and (927bp,204.65bp)  .. (927bp,185.67bp);
  % Edge: 5 -> 11
  \draw [->] (702.29bp,223.09bp) .. controls (698.04bp,214.36bp) and (692.85bp,203.68bp)  .. (683.63bp,184.75bp);
  % Edge: 7 -> 16
  \draw [->] (99bp,148.33bp) .. controls (99bp,140.26bp) and (99bp,130.65bp)  .. (99bp,111.67bp);
  % Edge: 2 -> 6
  \draw [->] (735.36bp,306.66bp) .. controls (773.95bp,293.43bp) and (849.24bp,267.64bp)  .. (902.73bp,249.31bp);
  % Edge: 8 -> 18
  \draw [->] (215.71bp,149.09bp) .. controls (219.96bp,140.36bp) and (225.15bp,129.68bp)  .. (234.37bp,110.75bp);
  % Edge: 4 -> 9
  \draw [->] (431.71bp,223.09bp) .. controls (435.96bp,214.36bp) and (441.15bp,203.68bp)  .. (450.37bp,184.75bp);
  % Node: 24
\begin{scope}
  \definecolor{strokecol}{rgb}{0.0,0.0,0.0};
  \pgfsetstrokecolor{strokecol}
  \draw (675bp,93bp) ellipse (27bp and 18bp);
  \draw (675bp,93bp) node {24};
\end{scope}
  % Node: 25
\begin{scope}
  \definecolor{strokecol}{rgb}{0.0,0.0,0.0};
  \pgfsetstrokecolor{strokecol}
  \draw (747bp,93bp) ellipse (27bp and 18bp);
  \draw (747bp,93bp) node {25};
\end{scope}
  % Node: 26
\begin{scope}
  \definecolor{strokecol}{rgb}{0.0,0.0,0.0};
  \pgfsetstrokecolor{strokecol}
  \draw (819bp,93bp) ellipse (27bp and 18bp);
  \draw (819bp,93bp) node {26};
\end{scope}
  % Node: 27
\begin{scope}
  \definecolor{strokecol}{rgb}{0.0,0.0,0.0};
  \pgfsetstrokecolor{strokecol}
  \draw (891bp,93bp) ellipse (27bp and 18bp);
  \draw (891bp,93bp) node {27};
\end{scope}
  % Node: 20
\begin{scope}
  \definecolor{strokecol}{rgb}{0.0,0.0,0.0};
  \pgfsetstrokecolor{strokecol}
  \draw (531bp,93bp) ellipse (27bp and 18bp);
  \draw (531bp,93bp) node {20};
\end{scope}
  % Node: 21
\begin{scope}
  \definecolor{strokecol}{rgb}{0.0,0.0,0.0};
  \pgfsetstrokecolor{strokecol}
  \draw (315bp,93bp) ellipse (27bp and 18bp);
  \draw (315bp,93bp) node {21};
\end{scope}
  % Node: 22
\begin{scope}
  \definecolor{strokecol}{rgb}{0.0,0.0,0.0};
  \pgfsetstrokecolor{strokecol}
  \draw (387bp,93bp) ellipse (27bp and 18bp);
  \draw (387bp,93bp) node {22};
\end{scope}
  % Node: 23
\begin{scope}
  \definecolor{strokecol}{rgb}{0.0,0.0,0.0};
  \pgfsetstrokecolor{strokecol}
  \draw (603bp,93bp) ellipse (27bp and 18bp);
  \draw (603bp,93bp) node {23};
\end{scope}
  % Node: 28
\begin{scope}
  \definecolor{strokecol}{rgb}{0.0,0.0,0.0};
  \pgfsetstrokecolor{strokecol}
  \draw (963bp,93bp) ellipse (27bp and 18bp);
  \draw (963bp,93bp) node {28};
\end{scope}
  % Node: 29
\begin{scope}
  \definecolor{strokecol}{rgb}{0.0,0.0,0.0};
  \pgfsetstrokecolor{strokecol}
  \draw (1035bp,93bp) ellipse (27bp and 18bp);
  \draw (1035bp,93bp) node {29};
\end{scope}
  % Node: 1
\begin{scope}
  \definecolor{strokecol}{rgb}{0.0,0.0,0.0};
  \pgfsetstrokecolor{strokecol}
  \draw (423bp,315bp) ellipse (27bp and 18bp);
  \draw (423bp,315bp) node {1};
\end{scope}
  % Node: 0
\begin{scope}
  \definecolor{strokecol}{rgb}{0.0,0.0,0.0};
  \pgfsetstrokecolor{strokecol}
  \draw (567bp,389bp) ellipse (27bp and 18bp);
  \draw (567bp,389bp) node {0};
\end{scope}
  % Node: 3
\begin{scope}
  \definecolor{strokecol}{rgb}{0.0,0.0,0.0};
  \pgfsetstrokecolor{strokecol}
  \draw (207bp,241bp) ellipse (27bp and 18bp);
  \draw (207bp,241bp) node {3};
\end{scope}
  % Node: 2
\begin{scope}
  \definecolor{strokecol}{rgb}{0.0,0.0,0.0};
  \pgfsetstrokecolor{strokecol}
  \draw (711bp,315bp) ellipse (27bp and 18bp);
  \draw (711bp,315bp) node {2};
\end{scope}
  % Node: 5
\begin{scope}
  \definecolor{strokecol}{rgb}{0.0,0.0,0.0};
  \pgfsetstrokecolor{strokecol}
  \draw (711bp,241bp) ellipse (27bp and 18bp);
  \draw (711bp,241bp) node {5};
\end{scope}
  % Node: 4
\begin{scope}
  \definecolor{strokecol}{rgb}{0.0,0.0,0.0};
  \pgfsetstrokecolor{strokecol}
  \draw (423bp,241bp) ellipse (27bp and 18bp);
  \draw (423bp,241bp) node {4};
\end{scope}
  % Node: 7
\begin{scope}
  \definecolor{strokecol}{rgb}{0.0,0.0,0.0};
  \pgfsetstrokecolor{strokecol}
  \draw (99bp,167bp) ellipse (27bp and 18bp);
  \draw (99bp,167bp) node {7};
\end{scope}
  % Node: 6
\begin{scope}
  \definecolor{strokecol}{rgb}{0.0,0.0,0.0};
  \pgfsetstrokecolor{strokecol}
  \draw (927bp,241bp) ellipse (27bp and 18bp);
  \draw (927bp,241bp) node {6};
\end{scope}
  % Node: 9
\begin{scope}
  \definecolor{strokecol}{rgb}{0.0,0.0,0.0};
  \pgfsetstrokecolor{strokecol}
  \draw (459bp,167bp) ellipse (27bp and 18bp);
  \draw (459bp,167bp) node {9};
\end{scope}
  % Node: 8
\begin{scope}
  \definecolor{strokecol}{rgb}{0.0,0.0,0.0};
  \pgfsetstrokecolor{strokecol}
  \draw (207bp,167bp) ellipse (27bp and 18bp);
  \draw (207bp,167bp) node {8};
\end{scope}
  % Node: 11
\begin{scope}
  \definecolor{strokecol}{rgb}{0.0,0.0,0.0};
  \pgfsetstrokecolor{strokecol}
  \draw (675bp,167bp) ellipse (27bp and 18bp);
  \draw (675bp,167bp) node {11};
\end{scope}
  % Node: 10
\begin{scope}
  \definecolor{strokecol}{rgb}{0.0,0.0,0.0};
  \pgfsetstrokecolor{strokecol}
  \draw (387bp,167bp) ellipse (27bp and 18bp);
  \draw (387bp,167bp) node {10};
\end{scope}
  % Node: 13
\begin{scope}
  \definecolor{strokecol}{rgb}{0.0,0.0,0.0};
  \pgfsetstrokecolor{strokecol}
  \draw (927bp,167bp) ellipse (27bp and 18bp);
  \draw (927bp,167bp) node {13};
\end{scope}
  % Node: 12
\begin{scope}
  \definecolor{strokecol}{rgb}{0.0,0.0,0.0};
  \pgfsetstrokecolor{strokecol}
  \draw (747bp,167bp) ellipse (27bp and 18bp);
  \draw (747bp,167bp) node {12};
\end{scope}
  % Node: 15
\begin{scope}
  \definecolor{strokecol}{rgb}{0.0,0.0,0.0};
  \pgfsetstrokecolor{strokecol}
  \draw (27bp,93bp) ellipse (27bp and 18bp);
  \draw (27bp,93bp) node {15};
\end{scope}
  % Node: 14
\begin{scope}
  \definecolor{strokecol}{rgb}{0.0,0.0,0.0};
  \pgfsetstrokecolor{strokecol}
  \draw (1035bp,167bp) ellipse (27bp and 18bp);
  \draw (1035bp,167bp) node {14};
\end{scope}
  % Node: 17
\begin{scope}
  \definecolor{strokecol}{rgb}{0.0,0.0,0.0};
  \pgfsetstrokecolor{strokecol}
  \draw (171bp,93bp) ellipse (27bp and 18bp);
  \draw (171bp,93bp) node {17};
\end{scope}
  % Node: 16
\begin{scope}
  \definecolor{strokecol}{rgb}{0.0,0.0,0.0};
  \pgfsetstrokecolor{strokecol}
  \draw (99bp,93bp) ellipse (27bp and 18bp);
  \draw (99bp,93bp) node {16};
\end{scope}
  % Node: 19
\begin{scope}
  \definecolor{strokecol}{rgb}{0.0,0.0,0.0};
  \pgfsetstrokecolor{strokecol}
  \draw (459bp,93bp) ellipse (27bp and 18bp);
  \draw (459bp,93bp) node {19};
\end{scope}
  % Node: 18
\begin{scope}
  \definecolor{strokecol}{rgb}{0.0,0.0,0.0};
  \pgfsetstrokecolor{strokecol}
  \draw (243bp,93bp) ellipse (27bp and 18bp);
  \draw (243bp,93bp) node {18};
\end{scope}
  % Node: 31
\begin{scope}
  \definecolor{strokecol}{rgb}{0.0,0.0,0.0};
  \pgfsetstrokecolor{strokecol}
  \draw (27bp,19bp) ellipse (27bp and 18bp);
  \draw (27bp,19bp) node {31};
\end{scope}
  % Node: 30
\begin{scope}
  \definecolor{strokecol}{rgb}{0.0,0.0,0.0};
  \pgfsetstrokecolor{strokecol}
  \draw (1107bp,93bp) ellipse (27bp and 18bp);
  \draw (1107bp,93bp) node {30};
\end{scope}
%
\end{tikzpicture}
% End of code




  \caption{Tree of cores for broadcast on gruyere}
  \label{fig:qrm_tree_gruyere}
\end{figure}

\paragraph{Observations} Reducing the number of children also helps in
select. It reduces the number of ``channels'' to poll, and therefore
the latency of detecting messages. Instead of

%%%%%%%%%%%%%%%%%%%%%%%%%%%%%%%%%%%%%%%%%%%%%%%%%%
\section{Experimental results}

\paragraph{Current numbers} Current numbers are listed in
Table~\ref{tab:bc_measurements}. 

\begin{table}[htb]
  \centering
  \begin{tabular}{lrrrr}
    \toprule
    broadcast algorithm & \multicolumn{2}{c}{nos5} & \multicolumn{2}{c}{gruyere} \\
      & cycles & error & cycles & error \\
    \midrule
    sequential &  4096.1 &  105.2 & 136751.9 &   3061.5 \\
    batch      &  2408.0 &  181.9 &  57360.8 &   5271.5 \\
    \bottomrule
  \end{tabular}
  \caption{Broadcast measurements}
  \label{tab:bc_measurements}
\end{table}

\paragraph{Measurements to do next} %
\begin{itemize}
\item Play with batching. Incrementally increase the number of
  messages send in a batch. I assume this is machine specific, which
  means, that this is something programmers do not want to deal
  with. We measure this at dynamically at run-time and find the right
  batch size.
\end{itemize}

\paragraph{Minimal cost of sending a tree-broadcast} %
Plot~\ref{pgfplot:201303141819} shows the cost of flooding a sub-tree
(with the 50\% worst measurements dropped). The group communication is
based on a binary tree. Core 0 is the root, cores 1 and 2 are its
children etc.

The average cost is really high (so something is wrong with my
code). But the minimal numbers show what is possible. The numbers
achieved are easily explainable. The cost is increasing
logarithmic with the number of nodes reached by the broadcast (as
expected). Every level in the tree adds an additional 3500 cycles to
the tree. Node 0 takes significantly longer. I don't know yet why that
is.

\begin{figure}
  \caption{Execution time for a broadcast with ACK on gruyere. The
    cost is for execution for the sub-graph starting at the given
    node. Refer to Figure~\ref{fig:qrm_tree_gruyere} for a
    visualization of the broadcast tree used. }
  \label{pgfplot:201303141819}
  \begin{tikzpicture}
    \begin{axis}[
      xlabel=core id,
      scaled y ticks = false, % prevent 10^x stuff
      y tick label style={/pgf/number format/fixed},
      ylabel={cost for subtree [cycles]}]
    \addplot[
      color=red,
      very thin,
      mark=*,
      mark options={%
        scale=.4
      },
      error bars/y dir=both,
      error bars/y explicit] coordinates {
      (0,9932.3) +- (564.4,564.4)
      (1,8240.1) +- (298.5,298.5)
      (2,8259.6) +- (549.4,549.4)
      (3,6206.3) +- (225.1,225.1)
      (4,5914.9) +- (399.2,399.2)
      (5,6033.5) +- (462.2,462.2)
      (6,6070.4) +- (434.3,434.3)
      (7,4530.4) +- (188.6,188.6)
      (8,3163.4) +- (246.1,246.1)
      (9,3403.0) +- (285.3,285.3)
      (10,3650.7) +- (336.5,336.5)
      (11,3559.0) +- (318.8,318.8)
      (12,3558.1) +- (284.8,284.8)
      (13,3567.5) +- (304.5,304.5)
      (14,3476.3) +- (290.2,290.2)
      (15,2711.7) +- (151.9,151.9)
      (16,614.0) +- (17.5,17.5)
      (17,606.2) +- (17.3,17.3)
      (18,636.2) +- (72.5,72.5)
      (19,641.3) +- (46.3,46.3)
      (20,583.8) +- (46.7,46.7)
      (21,577.8) +- (38.3,38.3)
      (22,561.8) +- (18.1,18.1)
      (23,603.5) +- (56.6,56.6)
      (24,589.0) +- (33.4,33.4)
      (25,582.9) +- (33.2,33.2)
      (26,571.7) +- (16.3,16.3)
      (27,581.6) +- (36.6,36.6)
      (28,564.0) +- (16.4,16.4)
      (29,564.0) +- (20.8,20.8)
      (30,551.6) +- (13.0,13.0)
      (31,596.2) +- (51.4,51.4)
    };
    \end{axis}
  \end{tikzpicture}

\end{figure}
  
\input{gruyere_bc_seq}

\begin{itemize}
\item Move server to a core != 0
\item Use low-level UMP stuff
{
\renewcommand{\labelitemi}{\checkmark}
\item disable yielding
}
\item run several request in parallel, give them IDs, can use in the
  measurement struct. Good for everything having transactions (DB,
  transactional memory, consistency in replication systems)
\end{itemize}

%%%%%%%%%%%%%%%%%%%%%%%%%%%%%%%%%%%%%%%%%%%%%%%%%%
\section{Future work}

\begin{itemize}
\item Need higher-level applications, such as:
  \begin{itemize}
  \item octopus
    \begin{itemize}
    \item higher level application?
    \end{itemize}
  \item capability system? \stefan{Ask Simon how expensive revocation
      currently is}
  \item a database with replication and consistency maintenance
    \begin{itemize}
    \item SharedDB? (buffers between operators, but no synchronization
      except for access to buffers, which are multiple writers, one
      reader)
    \item Crescando (some kind of state machine replication)
    \end{itemize}
  \end{itemize}
\item Machine model for topology aware tree
  \begin{itemize}
  \item Talked to Pravin about this a bit
  \end{itemize}
\item MPI collectives (as in~\cite{Tu2008})
\end{itemize}

%%%%%%%%%%%%%%%%%%%%%%%%%%%%%%%%%%%%%%%%%%%%%%%%%%
\section{Open questions}

\paragraph{How does this work with several senders} 
Maybe something like in navigation systems. Every node has an entry
point, which forwards the messages. Nodes are then clustered
\ldots. Or: send the request up to the root and then down again? This
keeps the message complexity the same and doubles the time
complexity. 

%%%%%%%%%%%%%%%%%%%%%%%%%%%%%%%%%%%%%%%%%%%%%%%%%%
\newpage
\bibliographystyle{plain}
\bibliography{defs,db,mendeley}

\label{LastPage}

\end{document}
