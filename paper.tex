\documentclass{article}
\usepackage{url,color,xspace,verbatim,subfig,ctable,multirow,listings}
\usepackage[utf8]{inputenc}
\usepackage[T1]{fontenc}
\usepackage{txfonts}
\usepackage{rotating}
\usepackage{paralist}
\usepackage{subfig}
\usepackage{graphics}
\usepackage{enumitem}
\usepackage{times}
\usepackage[colorlinks=true]{hyperref}

% ==================================================

\graphicspath{{figs/}}
\urlstyle{sf}

% tikz stuff
\usepackage{tikz}
\usetikzlibrary{shapes,positioning,calc}

\lstset{
  language=C,
  basicstyle=\ttfamily \small,
  flexiblecolumns=false,
  basewidth={0.5em,0.45em},
  boxpos=t,
}

\definecolor{skRed}{RGB}{155,25,25}
\newcommand{\stefan}[1]{
  {\color{skRed}[{\color{red}{SK}} #1]}}

% ==================================================

\begin{document}

\title{Replication on multicore considering hierarchy and
  characteristics of the network}

% email address
\newcommand{\eaddr}{stefan.kaestle@inf.ethz.ch}
\newcommand{\email}{\href{mailto:\eaddr}{\eaddr}}

\author{Stefan Kaestle\\
  \email \\
  Systems Group, ETH Zurich}

\maketitle

\paragraph{What do you want to enable?} Increase algorithm performance
on multicores by applying distributed principles, particularly
replication. Consistency using hierarchical group communication.

\paragraph{What problem are you solving, and why is it hard?} Three problems:
\begin{description}
\item[placement] machine dependent
\item[consistency] need to consider machine \emph{characteristics};
  multicores are \emph{different} from classical distributed systems.
\item[machine characteristics] diverse, fast changing, complex
\end{description}

\paragraph{What's the related work?} People do replication to deal
with scalability challenges on multicores, e.g.\
databases~\cite{Salomie2011, Wiesmann2000} and operating
systems~\cite{fos:osr09, tornado:osdi99, barrelfish:sosp09}. But they
don't to it properly. They do not consider machine characteristics.

\paragraph{What new idea(s) will solve the problems?} Consider machine
characteristics and keep in mind, that multicore machines are
different. \\
{\footnotesize Example: sequential consistency on shared memory, do
  not really need to have primary backup.}

\paragraph{How will you go about it?} Replicas on NUMA domains, show
that overall throughput is higher. Then, use a quorum based approach for
consistency. Consistency by doing a hierarchical group communication
approach instead of sequential send from same nodes.

\paragraph{How will you know and show it works?} Database?
Microbenchmark? Compare to primary backup, which is typically
used. Measure time required to apply an update.

\paragraph{What is your hypothesis?} We can do much better if machine
characteristics are considered.


%%%%%%%%%%%%%%%%%%%%%%%%%%%%%%%%%%%%%%%%%%%%%%%%%%
\newpage
\bibliographystyle{plain}
\bibliography{defs,db,mendeley}

\label{LastPage}

\end{document}
